\documentclass[
    ngerman,american
    ]{scrartcl}

    % ##########################################
    % # Choose the language for the document by editing below line
    % # de = German
    % # en = English
    \newcommand{\lang}{en}
    % ##########################################

    \usepackage{babel}
    \usepackage[utf8]{inputenc} 
    \usepackage{csquotes}
    \usepackage{enumitem}
    \usepackage{ifthen}
    \usepackage{lipsum}
    
    \newcommand{\paperSubTitle}[1]
{
    \ifthenelse{\equal{#1}{en}}{}{}
    \ifthenelse{\equal{#1}{de}}{Outline und Themenvorschlag}{}
}

\newcommand{\sectionQuestions}[1]
{
    \ifthenelse{\equal{#1}{en}}{\section{Introduction of the problem}}{}
    \ifthenelse{\equal{#1}{de}}{\section{Ziel der Arbeit - 4 Fragen}}{}
}

\newcommand{\sectionQuestionsDescription}[1]
{
    \ifthenelse{\equal{#1}{en}}{In this section the essence of the proposed work is described by answering several key questions. }{}
    \ifthenelse{\equal{#1}{de}}{Im Folgenden wird der Kern der Arbeit beschrieben indem vier Kernfragen beantwortet werden.}{}
}

\newcommand{\sectionInitialTOC}[1]
{
    \ifthenelse{\equal{#1}{en}}{\section{Preliminary notation and background}}{}
    \ifthenelse{\equal{#1}{de}}{\section{Vorläufige Gliederung}}{}
}

\newcommand{\sectionInitialTOCDescription}[1]
{
    \ifthenelse{\equal{#1}{en}}{We introduce some notations }{}
    \ifthenelse{\equal{#1}{de}}{Im Folgenden wird ein Inhaltverzeichnis für die vorgeschlagene Arbeit vorgestellt.}{}
}

\newcommand{\sectionSource}[1]
{
    \ifthenelse{\equal{#1}{en}}{\section{Relevant Related Work}}{}
    \ifthenelse{\equal{#1}{de}}{\section{Relevante verwandte Arbeiten}}{}
}


\newcommand{\sectionSourceDescription}[1]
{
    \ifthenelse{\equal{#1}{en}}{In this section, identified related work is described.}{}
    \ifthenelse{\equal{#1}{de}}{Diese Section stellt verwandte Arbeiten dar und erklärt kurz deren Bedeutung für die vorgeschlagene Arbeit.}{}
}

\newcommand{\questionOne}[1]
{
    \ifthenelse{\equal{#1}{en}}{Main questions related to our work\\}{}
    \ifthenelse{\equal{#1}{de}}{Was ist das Problem, welches Sie in Ihrer Arbeit bearbeiten wollen?}{}
}

\newcommand{\questionTwo}[1]
{
    \ifthenelse{\equal{#1}{en}}{Why is it a problem?}{}
    \ifthenelse{\equal{#1}{de}}{Warum ist es ein Problem?}{}
}

\newcommand{\questionThree}[1]
{
    \ifthenelse{\equal{#1}{en}}{Solution to the problem\\}{}
    \ifthenelse{\equal{#1}{de}}{Was ist die Lösung die sie entwickelt haben?}{}
}

\newcommand{\questionFour}[1]
{
    \ifthenelse{\equal{#1}{en}}{Why is it a solution?}{}
    \ifthenelse{\equal{#1}{de}}{Warum ist es eine Lösung?}{}
}


    \ifthenelse{\equal{en}{\lang}}
    {
        \selectlanguage{american} 
    }{
        \ifthenelse{\equal{de}{\lang}}
        {
            \selectlanguage{ngerman}
        }
        {\selectlanguage{american}}        
    }

    \usepackage[
        bibencoding=utf8, 
        style=alphabetic
    ]{biblatex}

    \bibliography{bibliography}
    
    
    \usepackage{amsmath}
    \title{
        % ##########################################
        % # Insert the title of your paper/thesis here
        % ###### 
        % Coming up with a good title is hard.
        % It should:
        %  1. capture the contents of the your work
        %  2. not be to broad or generic
        %  3. stick to the truth and don't not oversell
        %  4. use established terms and wordings
        %  5. make people curious about your work
        %  6. use current buzzwords if possilbe (but do it right)
        %  7. not use too many buzzwords :-)
        Research Proposal——Add privacy protection in Generative Adversial Networks(GAN)
        % ##########################################
        \\  \Large{\paperSubTitle{\lang}}} % don't touch this line

    \author{
        % ##########################################
        % # Your name goes here
        % ######
        % wWll, that should be obvious, right? 
        Haihan Gao,SA22011017\\
        Xiaolu Chen,SA22221005\\
        Li Zhang,SA22221064
        % ##########################################
        }
    
    \begin{document}
    
    % \bibliographystyle{plain}
      \maketitle
        \begin{abstract}

            \ifthenelse{\equal{en}{\lang}}{}{}
            % ##########################################
            % # Include your Abstract here 
            % ######
            % I would strongly suggest to start working on the abstract only 
            % after you have answered the 4 questions in Section 1, as this will
            % make it much easier for you to come up with an abstract that
            % is to the point, short, and still summarizing all the most crucial 
            % results of your work.
            %
            % The abstract should include the following points:
            %  - a short but to the point introduction of the problem area
            %  - what is the topic/problem, tackled in your work? 
            %  - why is the topic/problem of your work relevant? Why should the 
            %    reader care about it?
            %  - what are the results/answers of your work?
            %  - how did you gain your results and what is their quality?
            %                %  
            % It should NOT be:
            %  - too long/verbose
            %  - too short
            % \lipsum[1-2]
            Generative Model has been used to generate fake data from the distribution of real data.
            Some impressive work has been done with data generation such as Generative Adversarial Network(GAN) and Variation Autoencoder(VAE).
            An universial idea of building a generative model is to map real data(such as images and texts) into latent variables which represent the information or cluster characteristic in latent space.
            Then we use latent variables from encoder as the parameter of the target distribution.Sampling from the target distribution will generate new data which retains features from origin data as well as contains differences compared with origin input. 
            In order to use Gradient descent to optimize deep model,we need to design loss function predently.Generally the loss function is made up with two parts.The first one is difference with input,such as mseloss.Another one is KL divergence between latent distribution and a priority assumption.
            If we give the first part a high priority,we will find that our model tends to "copy" input data into output data.This may result into leakage of training data.
            % ##########################################

        \end{abstract}
        
        
        \sectionQuestions{\lang}
        
        \sectionQuestionsDescription{\lang}
        
        \begin{description}[style=unboxed]
            \item [\questionOne{\lang}] \vspace{\baselineskip}
                % ##########################################
                % # Question 1: What is the problem you want to address in your work? / 
                %               Was ist das Problem, welches Sie in Ihrer Arbeit bearbeiten wollen?
                % ######
                % The goal of this question is to clearly state what your work is about. 
                % What is the problem it is supposed to solve?
                %
                % Answering this question is particular important during the early phases 
                % of your work, in order to gain further insight and understanding about what 
                % your work is going to cover and address.  
                %
                % Answer this question very briefly by stating the problem or research 
                % question that you want to address/solve in your work.
                % 
                % Your answer should: 
                %  - only be 1 sentence (2 sentences max)
                %  - not cover a statement why the topic is relevant 
                %    for the industry (this is address by the next question)
                %  - properly use common terms and buzzwords of IT today (similar to the 
                %    rules for the abstract)
                %
                % Please acknowledge: the answer to this question should not cover why the 
                % problem is important or relevant to anyone (e.g. industry). This will 
                % be addressed with the next question.
                


                Machine Learning Tasks,including regression,classification and other ML task
                ,require a large amount of Training Data.Generally we collect data from real world.
                In some specific scenarios,data is expensive and contains some privacy informations,such as human face and fingerprint.
                An alternative solution to this problem is to build a model which adds noise with real data.This model is called generaitve model.
                However,there is a tradeoff between data diversity and data utilization.Diversity means generative data should be different from real data.Suppose a extreme scenarios,
                if the generative model only copy input images to output images,there will be little significance to introduce generative model.Utilization means the output should be able to  
                train downstream model in order to achieve our goals,such as image classification.
                % ##########################################

            % \item [\questionTwo{\lang}]
                % ##########################################
                % # Question 2: Why is it a problem? / Warum ist es ein Problem?
                % ######
                % The goal of this question is to describe why your work is relevant. 
                % Why should the reader care? Why is this the problem (of question 1)
                % worth investigating?
                %
                % Answering this question is particular important during the early phases 
                % of your work, in order to gain further insight and understanding of the 
                % problem domain you are addressing. Further, it is a good checkpoint to 
                % ensure that you are addressing issues that are not just theoretical but 
                % have real-world applications. 
                %
                % Your answer should: 
                %  - be 3 - 5 sentences 
                %  - give a broader overview of the domain/area where your problem occurs
                %  -- who has this problem?
                %  -- what is the impact of it?
                %  -- which conditions need to be fulfilled for the problem to occur?
                %  -- etc.
                %  - describe the benefit of having the problem resolved
                % insert answer here
                % ##########################################

            \item [\questionThree{\lang}]
                % ##########################################
                % # Question 3: What is the solution you developed in your work? / 
                %               Was ist die Lösung die sie entwickelt haben?
                % ######
                % The goal of this question is to describe the results of your work 
                % ans/or solution to the problem of your work.
                % 
                % It is hard/impossible to answer this question in the early phases of 
                % your work, as usually you do not have results, yet. However, you can 
                % already state first ideas that you may have in order to discuss them 
                % with your supervisor. 
                %
                % Your answer should:
                %  - clearly state all results of your work, that are relevant to your 
                %    research problem. 
                %  - not oversell your results, stick you what you actually have 
                %    accomplished"
                %  - give credit where credit is due. If you created your results based 
                %    on the work of others, give them the credit.
                %  - if none of your ideas did not produce any usable solution, state 
                %    so - these attempts are also results! By documenting them, it may 
                %    prevent others from trying them as well. 
                % insert answer here
                % ##########################################
                \par
                Based on the work of differential privacy in deep learning,we add noise in gradient
                when updating parameter of networks.GAN is consisted with Generative Net and Discrimination
                Net and they are trained with a max-min loss.\\
                According to the structure of GAN,the Discrimination net is removed after the whole net is well trained.
                 only add noise when gradient is passed from Discrimination to Generative net.This well help us set better noise range $\epsilon$


            %\item [\questionFour{\lang}]
                % ##########################################
                % # Question 4: Why is it a solution? / Warum ist es eine Lösung?
                % ######
                % The goal of this question is to describe who you developed your results 
                % and what the quality of them are.
                % 
                % Your answer should:
                %  - short and to the point
                %  - clearly state how you developed your results
                %  -- what is the chain of reasoning that led to your results/solution
                %  -- what statistics, literature, studies, or other literature did you 
                %     base your assumptions on?
                %  - clearly state the quality and applicability of your results
                %  -- reflect your work objectively - there is no perfection in this 
                %     world, so your work is not perfect as well. Be aware of that!
                %  -- how did you ensure that your results are accurate? did you:
                %  --- perform experiments? 
                %  --- apply any logical deductions?
                %  --- mathematical proofs? 
                %  --- implement a "proof of concept" implementation and evaluate? 
                %  --- etc.
                %  - clearly state the shortcomings of your work
                %  -- be hones and objective about your own work. 
                %  -- In which cases/scenarios are your results applicable?
                % include answer here
                % ##########################################
        \end{description}
        
        \sectionInitialTOC{\lang}
        % \sectionInitialTOCDescription{\lang}
        Suppose $\mathcal F\subset \mathcal R^n$ represents data containing some semantic information,such as human hand-writing numbers.
        Training data set $f$ is a subset of $\mathcal F$.
        Now we want to generate another $q\in \mathcal R^n$ from noise $p\in \mathcal R^m$ that has similar semantic information with vector in $\mathcal F$.
        GAN is composed with two networks:1. Generative Network $\mathcal G:\mathcal R^m\to \mathcal R^n$ and 2. Discriminator Network $\mathcal D:\mathcal R^n \to (0,1)$\\
        Discriminator network $\mathcal D$ is used to identify the difference between generative data and real data.Its output is normalized from 0 to 1 to represent the possibility that input data is from real dataset rather than from generator  
        \subsection{Train process of GAN}
        It is consisted with two processes of training.
        \subsubsection{Make $\mathcal D$ be able to identify real data}
        Sample k data from real dataset $f_1,f_2,\cdots,f_k\in f$ and send them to Discriminator $\mathcal D$.We want $\mathcal D$ differentiate them to be data with proper semantic information.
        \begin{equation}
            \max_{f^\prime \in f}E[D(f^\prime)]
        \end{equation}
        \subsubsection{Make $\mathcal D$ be able to identify fake data}
        Now sample k random vector from $\mathcal R^n$,denoted as $v_1,v_2,\cdots,v_k$.Send them into Generator $\mathcal G$,we have $\hat v_i = G(v_i)$.We want our Discriminator $\mathcal D$ could differentiate $\hat v_i$ from real dataset.
        \begin{equation}
            \min_{v\in \mathcal R^n} E[G(v)]
        \end{equation}
        
        % DP-SGD. 
% PATE. 
% Fed-Avg GAN.
% 3.5.	Preliminary scheme
% We use a new gradient-sanitized Wasserstein GAN (GS-WGAN), which is capable of generating high-dimensional data with DP guarantees, in both centralized and decentralized datasets. In the decentralized case, user-level DP security is guaranteed under an untrusted server. And we evaluate our method on various datasets against other state-of-the-art approaches.
 
        % ##########################################
        % # Proposed table of contents
        % ######
        % The goal of this section is to propose a table of contents. Please keep in
        % mind: a well created table of contents is very powerful in provide a good 
        % overview of the overall chain of reasoning of your work. This makes it 
        % extremely valuable.
        % 
        % Please include:
        %  - names of sections and subsections (please don't go deeper than that unless 
        %    your supervisor asks you for it)
        %  - a brief description of the proposed content of each section and subsection
        %    (1-3 sentences)
        % 
        % \begin{enumerate}
        %     \item \textbf{Section 1 Name} insert brief description
        %             \begin{enumerate}
        %                 \item \textbf{Subsection 1 Name} insert brief description
        %                 \item \textbf{Subsection 2 Name} insert brief description
        %             \end{enumerate}
        %     \item \textbf{Section 2 Name} insert brief description
        % \end{enumerate}
        % ##########################################
    
      
        \sectionSource{\lang}
        \sectionSourceDescription{\lang}
        \begin{enumerate}
            \item{DP-SGD}
            Training GANs with the DP-SGD method can be effective in generating high-dimensional sanitized data. However, DP-SGD relies on the clipping bound of gradient norm, i.e., the sensitivity value. Sensitivity values vary greatly with model architecture and training dynamics, which makes the implementation of DP-SGD difficult.
            \item{PATE} 
            Privatization of Teacher Aggregates (PATE) has recently been adapted to generative models. and two main approaches were studied: PATE-GAN and G-PATE. Both of these methods only train generators with DP guarantees, however, the gradients of G-PATE need to be manually selected to adapt to the PATE framework and secondly the high-dimensional nature of the PATE framework brings high privacy costs.
            \item{Fed-Avg GAN}
             Fed-Avg GAN is a solution to the decentralized case by using the DP-Fed-Avg algorithm to adapt GAN training to provide user-level DP guarantees under a trusted server. And it merely works on decentralized data.
        \end{enumerate}
        \begin{thebibliography}{99}
            \bibitem{ref1} Abadi, Martin and Chu, Andy and Goodfellow, Ian and McMahan, H Brendan and Mironov, Ilya and Talwar, Kunal and Zhang, Li,Deep learning with differential privacy,Proceedings of the 2016 ACM SIGSAC conference on computer and communications security
            \bibitem{ref2} Chen D, Orekondy T, Fritz M. Gs-wgan: A gradient-sanitized approach for learning differentially private generators[J]. Advances in Neural Information Processing Systems, 2020, 33: 12673-12684.

        \end{thebibliography}

        % \section{related work}
    %     % \begin{thebibliography}{99}  
    %     %     \bibitem{ref1}Zheng L, Wang S, Tian L, et al., Query-adaptive late fusion for image search and person re-identification, Proceedings of the IEEE Conference on Computer Vision and Pattern Recognition, 2015: 1741-1750.  
    %     %     \bibitem{ref2}Arandjelović R, Zisserman A, Three things everyone should know to improve object retrieval, Computer Vision and Pattern Recognition (CVPR), 2012 IEEE Conference on, IEEE, 2012: 2911-2918.  
    %     %     \bibitem{ref3}Lowe D G. Distinctive image features from scale-invariant keypoints, International journal of computer vision, 2004, 60(2): 91-110.  
    %     %     \bibitem{ref4}Philbin J, Chum O, Isard M, et al. Lost in quantization: Improving particular object retrieval in large scale image databases, Computer Vision and Pattern Recognition, 2008. CVPR 2008, IEEE Conference on, IEEE, 2008: 1-8.  
    %     % \end{thebibliography}
    % \bibliography{bibfile}
        % ##########################################
        % # Overview of identified relevant work
        % ######
        % The goal of this section is to provide an overview of the relevant and significant 
        % related work identified so far. Make sure that your cited sources are of appropriate
        % quality!
        % 
        % Please include:
        %  - a citation of the source using Latex facilities (incl. a generated list of 
        %    references)
        %  - a brief descriptions of the source and a statement why this is relevant for 
        %    your work (1-2 sentences)
        % 
        % \begin{description}
        % \item[\cite{gruba_how_2017}] insert brief description
        % \item[\cite{zobel_writing_2015}] insert brief description
        % \end{description}
        % ##########################################
        
        
      \printbibliography
    
    \end{document}